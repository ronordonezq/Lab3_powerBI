\section{Desarrollo } 


 \subsection {EJERCICIO 1}

\begin{itemize}
\item Tarea 1 : Conectar a datos existentes


\begin{enumerate}
\item Abrir SQL Server Management Studio y conectar a la intancia de base de datos (local) utilizando autenticacion de Windows ,a brir la solucion Project.ssmssln , cuando carge expandir las Conultas (Queries) y hacer click en Exercise1.sql

\begin{center}
\includegraphics[scale=0.55]{./Imagenes/tarea1_1.png}
\end{center}

\item Abrir PowerBI Desktop y conectar con la base de datos AdventureWorksLT2016 . Seguidamente expandir las opciones avanzadas y copiar el Task 1 del Lab Exercise 1.sql en la descripcion. 


\begin{center}
\includegraphics[scale=0.55]{./Imagenes/tarea1_2.png}
\end{center}

\begin{center}
\includegraphics[scale=0.55]{./Imagenes/tarea1_3.png}
\end{center}
\end{enumerate}

\begin{center}
\includegraphics[scale=0.55]{./Imagenes/tarea1_final.png}
\end{center}

\item Tarea 2 : Formar Datos


\begin{enumerate}
\item Renombrar ambas consultas creadas de la Tarea 1 y renombrar el primero Query como Customers .De la misma manera con el Query2 como Sales
\begin{center}
\includegraphics[scale=0.55]{./Imagenes/tarea1_4.png}
\end{center}




\item De la tabla Customers eliminar las filas NameStyle , SalesPerson . Tambien en las filas AddressLine1, City , StateProvince ,ContryRegion y PostalCode hacer click en Modelado  y en Categoria de datos : Sin clasificar 	

\begin{center}
\includegraphics[scale=0.55]{./Imagenes/20.png}
\end{center}

\begin{center}
\includegraphics[scale=0.55]{./Imagenes/21.png}
\end{center}

\begin{center}
\includegraphics[scale=0.55]{./Imagenes/22.png}
\end{center}

\begin{center}
\includegraphics[scale=0.55]{./Imagenes/23.png}
\end{center}

\begin{center}
\includegraphics[scale=0.55]{./Imagenes/24.png}
\end{center}

\item Agregar una nueva Columna en la parte de Calculos en nueva columna FullAddress.

\begin{center}
\includegraphics[scale=0.55]{./Imagenes/tarea2_formula.png}
\end{center}

\item En la tabla Sales , eliminar as filas RevisionNumer y SalesOrderNumer. Tambien ocultar en la vista de informes las columnas CustomerID, SalesOrderID, SalesOrderDetailID
\begin{center}
\includegraphics[scale=0.55]{./Imagenes/29.png}
\end{center}

\item Agregar una nueva Columna en la parte de Calculos en nueva columna LineTotal

\begin{center}
\includegraphics[scale=0.55]{./Imagenes/tarea2_formula2.png}
\end{center}




\end{enumerate}



\item Tarea 3 : Combinar Datos


\begin{enumerate}
\item Abrir el archivo State.xlsx y hacer Ctrl+ C a la tabla . Seguidamente Abrir PowerBI y en el cuadro de Datos externos hacer click en Especificar Datos y hacer Ctrl+ V 

\begin{center}
\includegraphics[scale=0.55]{./Imagenes/tarea3_creartabla.png}
\end{center}

\begin{center}
\includegraphics[scale=0.55]{./Imagenes/tarea3_tabla.png}
\end{center}

\item Renombrar la tabla creada a Sales by State.

\begin{center}
\includegraphics[scale=0.55]{./Imagenes/tarea3_cambiarnombre.png}
\end{center}

\item Despues abrir en PowerBI en Obtener Datos y en Web , poniendo la siguiente URL:

\begin{center}
\includegraphics[scale=0.55]{./Imagenes/tarea3_URL.png}
\end{center} 

\item Seguidamente escoger la tabla Codes and abbreviations for U.S. states, territories and other regions y cargar datos . Hacer click en Editar Consultas . En la ventana emergente hacer click en Reducir Filas despues en Quitar Filas y en Quitar filas inferiores , especificando las 26 ultimas filas en la ventana emergente.

\begin{center}
\includegraphics[scale=0.55]{./Imagenes/tarea3_removerows.png}
\end{center}

\begin{center}
\includegraphics[scale=0.55]{./Imagenes/tarea3_26rows.png}
\end{center}


\item 

\begin{center}
\includegraphics[scale=0.55]{./Imagenes/24.png}
\end{center}

\item Agregar una nueva Columna en la parte de Calculos en nueva columna y copiar la siguiente formula

\begin{center}
\includegraphics[scale=1]{./Imagenes/tarea2_formula.png}
\end{center}

\item En la tabla Sales , eliminar as filas RevisionNumer y SalesOrderNumer. Tambien ocultar en la vista de informes las columnas CustomerID, SalesOrderID, SalesOrderDetailID
\begin{center}
\includegraphics[scale=0.55]{./Imagenes/29.png}
\end{center}

\item Agregar una nueva Columna en la parte de Calculos en nueva columna y copiar la siguiente formula

\begin{center}
\includegraphics[scale=1]{./Imagenes/tarea2_formula2.png}
\end{center}

\item Hacer click en LineTotal	 y en el menu de herrameintas superior  en Modelado en cuadro de Formato , hacer click en Formato:General y en Moneda colocar English (United States) y Guardar lo avanzado.

\item Seleccionar las columnas de ASNI2 , AP, GPO, Header, Name and status of region2, Other abbreviations , USCG y USPS y eliminarlas. Seguidamente en el cuadro de Consulta en Propiedades , en Nombre poner States with Code y Aceptar.
\item Despues  en el cuadro de Transformar de menu , hacer click en Usar la primera fila como encabezado de tal manera que Renombrando United States of America por State Name , US USA 840 por State Code Long y US por State Code Short y Aceptar. Debe aparecer de esta manera la tabla generada:
\begin{center}
\includegraphics[scale=1]{./Imagenes/tarea3_renombrartablas.png}
\end{center}

\item En el cuadro  Combinar , hacer click en Combinar Consultas . En la ventana emergente de la tabla Sales by State seleccionar la columna States y en la iguiente tabla escoger States with Code y seleccionar la columna StateName y aceptar . Renombrar el nombre de la fila  nueva a State Code y expandir , seleccionando solo State Code Short y Aceptar.

\begin{center}
\includegraphics[scale=1]{./Imagenes/tarea3_tablacombinada.png}
\end{center}

\item Como ultimo paso Ocultar de ls vista de informes la tabla States with Code.
\begin{center}
\includegraphics[scale=1]{./Imagenes/tarea3_ocultartabla.png}
\end{center}
\end{enumerate}


\subsection {EJERCICIO 2 : Construyendo Reportes en PowerBI}

\begin{enumerate}
\item Tarea 1 : Crear un Grafico

\begin{center}
\includegraphics[scale=0.66]{./Imagenes/ejer2_tarea1.png}
\end{center}

\item Tarea 2: Crear una Visualizaci\'on de Mapa a Map Visualization

\begin{center}
\includegraphics[scale=0.70]{./Imagenes/ejer2_tarea2.png}
\end{center}

\end{enumerate}


\subsection {EJERCICIO 3 : Creando un Panel en PowerBI}
\begin{enumerate}

\item Tarea 1 :Publicar todo lo avanzado hasta ahora . Asegurar de iniciar correctamente las sesiones solicitadas.

\begin{center}
\includegraphics[scale=0.66]{./Imagenes/ejer3_publicar.png}
\end{center}

\item Tarea 2: Crear una Visualizaci\'on de Mapa a Map Visualization : En Informes, haga clic en Ventas de AdventureWorksLT y seguidamente por cada grafico creado anclar y agregar el titulo igual que el creado y un subtitulo.

\begin{center}
\includegraphics[scale=0.70]{./Imagenes/ejer3_clickBD.png}
\end{center}

\begin{center}
\includegraphics[scale=0.60]{./Imagenes/ejer3_panel1.png}
\end{center}
\begin{center}
\includegraphics[scale=0.55]{./Imagenes/ejer3_panel2.png}
\end{center}

Ruta del informe en el Power BI
\\
\url{https://app.powerbi.com/groups/me/reports/49d0eea9-f1a5-4edb-bb43-3a563dbd5390?ctid=b6b466ee-468d-4011-b9fc-fbdcf82ac90a} 

\end{enumerate}

\end{itemize}







